\documentclass[journal,12pt,twocolumn]{IEEEtran}

\usepackage{setspace}
\usepackage{gensymb}
\singlespacing
\usepackage[cmex10]{amsmath}

\usepackage{amsthm}
\usepackage{tcolorbox}
\usepackage{mathrsfs}
\usepackage{txfonts}
\usepackage{stfloats}
\usepackage{bm}
\usepackage{cite}
\usepackage{cases}
\usepackage{subfig}

\usepackage{longtable}
\usepackage{multirow}

\usepackage{enumitem}
\usepackage{mathtools}
\usepackage{steinmetz}
\usepackage{tikz}
\usepackage{circuitikz}
\usepackage{verbatim}
\usepackage{tfrupee}
\usepackage[breaklinks=true]{hyperref}
\usepackage{graphicx}
\usepackage{tkz-euclide}

\usetikzlibrary{calc,math}
\usepackage{listings}
    \usepackage{color}                                            %%
    \usepackage{array}                                            %%
    \usepackage{longtable}                                        %%
    \usepackage{calc}                                             %%
    \usepackage{multirow}                                         %%
    \usepackage{hhline}                                           %%
    \usepackage{ifthen}                                           %%
    \usepackage{lscape}     
\usepackage{multicol}
\usepackage{chngcntr}

\DeclareMathOperator*{\Res}{Res}

\renewcommand\thesection{\arabic{section}}
\renewcommand\thesubsection{\thesection.\arabic{subsection}}
\renewcommand\thesubsubsection{\thesubsection.\arabic{subsubsection}}

\renewcommand\thesectiondis{\arabic{section}}
\renewcommand\thesubsectiondis{\thesectiondis.\arabic{subsection}}
\renewcommand\thesubsubsectiondis{\thesubsectiondis.\arabic{subsubsection}}


\hyphenation{op-tical net-works semi-conduc-tor}
\def\inputGnumericTable{}                                 %%

\lstset{
%language=C,
frame=single, 
breaklines=true,
columns=fullflexible
}
\begin{document}

\setlength{\parskip}{1em}
\newtheorem{theorem}{Theorem}[section]
\newtheorem{problem}{Problem}
\newtheorem{proposition}{Proposition}[section]
\newtheorem{lemma}{Lemma}[section]
\newtheorem{corollary}[theorem]{Corollary}
\newtheorem{example}{Example}[section]
\newtheorem{definition}[problem]{Definition}

\newcommand{\BEQA}{\begin{eqnarray}}
\newcommand{\EEQA}{\end{eqnarray}}
\newcommand{\define}{\stackrel{\triangle}{=}}
\bibliographystyle{IEEEtran}
\raggedbottom
\setlength{\parindent}{0pt}
\providecommand{\mbf}{\mathbf}
\providecommand{\pr}[1]{\ensuremath{\Pr\left(#1\right)}}
\providecommand{\qfunc}[1]{\ensuremath{Q\left(#1\right)}}
\providecommand{\sbrak}[1]{\ensuremath{{}\left[#1\right]}}
\providecommand{\lsbrak}[1]{\ensuremath{{}\left[#1\right.}}
\providecommand{\rsbrak}[1]{\ensuremath{{}\left.#1\right]}}
\providecommand{\brak}[1]{\ensuremath{\left(#1\right)}}
\providecommand{\lbrak}[1]{\ensuremath{\left(#1\right.}}
\providecommand{\rbrak}[1]{\ensuremath{\left.#1\right)}}
\providecommand{\cbrak}[1]{\ensuremath{\left\{#1\right\}}}
\providecommand{\lcbrak}[1]{\ensuremath{\left\{#1\right.}}
\providecommand{\rcbrak}[1]{\ensuremath{\left.#1\right\}}}
\theoremstyle{remark}
\newtheorem{rem}{Remark}
\newcommand{\sgn}{\mathop{\mathrm{sgn}}}
% \providecommand{\abs}[1]{\left\vert#1\right\vert}
% \providecommand{\res}[1]{\Res\displaylimits_{#1}} 
% \providecommand{\norm}[1]{\left\lVert#1\right\rVert}
% %\providecommand{\norm}[1]{\lVert#1\rVert}
% \providecommand{\mtx}[1]{\mathbf{#1}}
% \providecommand{\mean}[1]{E\left[ #1 \right]}
\providecommand{\fourier}{\overset{\mathcal{F}}{ \rightleftharpoons}}
%\providecommand{\hilbert}{\overset{\mathcal{H}}{ \rightleftharpoons}}
\providecommand{\system}{\overset{\mathcal{H}}{ \longleftrightarrow}}
	%\newcommand{\solution}[2]{\textbf{Solution:}{#1}}
\newcommand{\solution}{\noindent \textbf{Solution: }}
\newcommand{\cosec}{\,\text{cosec}\,}
\providecommand{\dec}[2]{\ensuremath{\overset{#1}{\underset{#2}{\gtrless}}}}
\newcommand{\myvec}[1]{\ensuremath{\begin{pmatrix}#1\end{pmatrix}}}
\newcommand{\mydet}[1]{\ensuremath{\begin{vmatrix}#1\end{vmatrix}}}
\numberwithin{equation}{subsection}
\makeatletter
\@addtoreset{figure}{problem}
\makeatother
\let\StandardTheFigure\thefigure
\let\vec\mathbf
\renewcommand{\thefigure}{\theproblem}
\def\putbox#1#2#3{\makebox[0in][l]{\makebox[#1][l]{}\raisebox{\baselineskip}[0in][0in]{\raisebox{#2}[0in][0in]{#3}}}}
     \def\rightbox#1{\makebox[0in][r]{#1}}
     \def\centbox#1{\makebox[0in]{#1}}
     \def\topbox#1{\raisebox{-\baselineskip}[0in][0in]{#1}}
     \def\midbox#1{\raisebox{-0.5\baselineskip}[0in][0in]{#1}}
\vspace{3cm}
\title{Assignment 1}
\author{Kartikeya Jaiswal - EE18BTECH11025}
\maketitle
\newpage
\bigskip
\renewcommand{\thefigure}{\theenumi}
\renewcommand{\thetable}{\theenumi}
Download all latex-tikz codes from 
%
\begin{lstlisting}
https://github.com/kartikeyajaiswal/c-ds
\end{lstlisting}
\section{Problem}
(Q 34) Each of a set of \emph{n} processes executes the following code using two semaphores \emph{a} and \emph{b} initialized to 1
and 0, respectively. Assume that \emph{count} is a shared variable initialized to 0 and not used in CODE SECTION P. 
\begin{lstlisting}
CODE SECTION P

Wait (a); count = count + 1;
if (count ==n) signal (b);
signal (a); wait (b); signal (b);

CODE SECTION Q
\end{lstlisting}
What does the code achieve?
\begin{enumerate}
    \item It ensures that no process executes CODE SECTION Q before every process has finished CODE SECTION P.
    \item It ensures that at most two processes are in CODE SECTION Q at any time.
    \item It ensures that all processes execute CODE SECTION P mutually exclusively.
    \item It ensures that at most n–1 processes are in CODE SECTION P at any time.
\end{enumerate}
\section{Solution}
Answer : (A) 
\indent It ensures that no process executes CODE SECTION Q before every process has finished CODE SECTION P.
\newline
\newline
\textbf{Explanation}
\newline
The definition of wait() function:
\begin{lstlisting}[language=C]
wait(Semaphore a)
{
   while(a<=0){
        ;      //no operation
   }
   a--;
}
\end{lstlisting}

and the definition of signal() function:
\begin{lstlisting}
signal(Semaphore a)
{
   a++;
}
\end{lstlisting}

Consider process 1 (say P1):
\begin{lstlisting}[language=C]
n;
count = 0; 
a = 1; 
b = 0; 

CODE SECTION P

Wait (a);           //a=0
count = count + 1;  //count = 1
if (count ==n) signal (b);      //since count!=n, b=0
signal (a);         // a=1
wait (b);           // no operation
signal (b);

CODE SECTION Q
\end{lstlisting}

In the last section, wait(b) doesn't execute, since wait() does no operation for semaphore = 0. This process P1 is stuck here, but \emph{count} is updated to count=1.
\newline
\newline
Now, similarly consider process P2:
\begin{lstlisting}[language=C]
n;
count = 1; 

CODE SECTION P

Wait (a);           //a=0
count = count + 1;  //count = 2
if (count ==n) signal (b);      //since count!=n, b=0
signal (a);         // a=1
wait (b);           // no operation
signal (b);

CODE SECTION Q
\end{lstlisting}
Here as well, wait(b) doesn't execute, since wait() does no operation for semaphore = 0. This process P1 is stuck here, but \emph{count} is updated to count=2.
\par

Clearly, for each process executed, the \emph{count} increases by 1.
So, at the end of (n-1)\textsuperscript{th} process, \emph{count} = n-1.
Consider the n\textsuperscript{th} process:

\begin{lstlisting}[language=C]
n;
count = n-1; 

CODE SECTION P

Wait (a);           //a=0
count = count + 1;  //count = n
if (count ==n) signal (b);      //since count=n, b=1
signal (a);         // a=1
wait (b);           // b=0
signal (b);         // b=1

CODE SECTION Q
\end{lstlisting}

Here, the wait(b) at the end executes (since count=n), 
so the n\textsuperscript{th} process completes here and its not stuck.
Here signal(b) completes execution changing value of b to b=1, 
so all the previously stuck processes will also complete execution.

So, option (A) is correct.

\end{document}